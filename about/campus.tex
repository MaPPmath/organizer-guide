Here are some guidelines for local campuses on how to prepare for and
run the event.

\phSection{Schedule Template}

\begin{itemize}
  \item 0:00 - Staff arrives
  \item 0:15 - Team check-in
  \item 0:45 - Orientation
  \item 1:00 - Game Begins
  \item 4:00 - Game Ends
  \item 4:15 - Wrap-Up and Awards
  \item 4:30 - Dismissal
\end{itemize}

\phSection{Volunteers}

Only a handful of volunteers are required to run Game Control. We recommend
having 2-5 volunteers depending on the number of participating teams.

\phSection{Classroom Space}

A large \textbf{lecture hall} is recommended for running Check-in,
Orientation, and the Wrap-Up. Game Control can be stationed there during
the game as well, or another nearby classroom.

Each team should be given a separate \textbf{classroom} so that they may
openly collaborate with teammates without spoiling puzzles for other teams.
It is useful to affix \textbf{printed signs} on each classroom and Game
Control to help players navigate your space, as well as any additional
signage required to get around.

\phSection{Team Supplies}

\textbf{Scissors and tape} should be provided in each classroom.
In addition, \textbf{chalk or whiteboard markers} should be provided if
teams will have access to chalkboards or whiteboards in their room.
Some campuses provide \textbf{pencils and notepads} to players.
We recommend \textbf{inviting teams to bring additional supplies}, such as
graph paper or colored pencils.

Note that teams may also choose to bring laptops, cameras,
and so on, and must provide their own \textbf{smart phones/devices}
unless your campus provides them instead. 
We discourage campuses from banning the use of computers or the internet,
since a phone can be used to perform the same tasks, but also
do not suggest to explicitly recommend such items as they aren't required
to enjoy or be competitive in this game. The puzzles are
designed so that they generally cannot be solved using Google or 
brute force methods.

\newpage

\phSection{Copies}

Generally, each game is designed to only require a printed copy
of the game book PDF and the ClueKeeper app.
The PDF is designed to be printed/copied in \textbf{grayscale}, both
for the convenience of campuses and for accessibility by players.
It is recommended to print
\textbf{at least one stapled, single-sided
copy of the game book per player, plus extras}
on your access to last-minute copying.

An alternative version of the PDF that includes flavortext from the
app and other useful information on the game will also be provided.
We recommend printing \textbf{one copy per volunteer} in a binder.
\textbf{One or more copies of this document} may also be useful.

\phSection{Registration and Check-in}

In the weeks before your event, you should offer
\textbf{online registration} for participating schools, perhaps via
Google Forms. You may choose to collect a registration fee to
offset costs to run the game locally.

Schools may not know exactly which or how many students will play
the game, so it's important to have a \textbf{check-in process}
on the day of the event to determine which students will play on which teams.

At check-in, \textbf{distribute printed player PDF ``Game Books''} - these puzzles
cannot be solved until the game is started in the app.
Each team should have access to \textbf{one or more devices with the ClueKeeper
app} (often the players' phones). Using a \textbf{Hunt Code} purchased
from ClueKeeper, one device for each team can download the hunt, and then
add teammate devices by using their chosen ClueKeeper username, or
Guest-XYZ username.

Each team is assigned a \textbf{Headquarters/HQ}, a classroom to use
as the home base for solving puzzles.

Some campuses also choose to distribute other giveaways/swag/brochures
at registration. Many bookstores are willing to provide branded disposable
bags to help distribute materials.

\phSection{Orientation}

At an Orientation gathering of all participating players,
the rules should be reviewed, and any questions from players
should be answered. In particular, boundaries for where players might
need to travel during the game (if GPS is used) should be established.

Once everyone is ready, present teams with the Start Code that will allow them
to begin solving puzzles and start the game timer. Teams should be instructed
to wait to enter it until they are settled in their HQ and ready to start solving.

\phSection{Gameplay}

As clues are unlocked in ClueKeeper, players will be able to solve puzzles
and input their solutions into ClueKeeper, earning points. Progress
may be monitored at ClueKeeper.com.

Campuses using Cluekeeper's GPS will have players traveling to locations
on campus to unlock Main Puzzles and Cryptic Puzzles. A campus map should
be provided mapping the numbers 000 to 999 to various locations on campus
(e.g. 000-049 is Building A, 050-099 is Building B, etc.). 
GPS enforcement can be turned off in case of inclement
weather, in which case the
three digit code may be entered from any location. An example of this map
for the University of South Alabama is included in this document; smaller
campuses may use less locations for a wider range of three-digit codes.

A volunteer should be available in a room (``Game Control'') to
provide support to teams and review any human-graded solutions.
Another volunter might stand at the door of this room to ensure at most
one team is allowed in at all times, to avoid accidental
spoiling of puzzle information between teams.

When the game includes a human-graded Bonus Puzzle,
each team is allowed three submissions.
The puzzle should be judged by Game Control in front of the players to confirm
the validity of the submission.
Players should be told to bring their app to Game Control to confirm their
game has not expired.

\phSection{Food}

Campuses that will be running the event through lunchtime are encouraged to
provide a \textbf{pizza lunch} for players. This lunch should not interrupt the
game; rather, players should be able to grab a bite to eat to have while they
continue to solve puzzles. In addition, \textbf{snacks}
(fruit, granola bars, etc.) and \textbf{drinks} (bottled water) are nice for
players to have access to during the game. Don't forget to provide
appropriate \textbf{plates, cutlery, napkins, and trashbags}.

This food can be distributed at a \textbf{central location near Game Control}
(but not inside Game Control's room where puzzles will be discussed).

\phSection{Wrap-Up and Awards}

At the end of the game, teams should straighten up their classrooms before
returning to Game Control for the Wrap-Up. \textbf{Trash bags} may be
provided for this purpose.

Teams should line up outside Game Control until results have been tabulated.
Once all results have been determined, teams may be seated inside Game Control.

Solutions need not be reviewed together; MaPP will provide video
content outlining the puzzle content to be viewed after the event.

\textbf{Certificates} might be distributed in
random or alphabetical order to all teams placing
below 3rd place.
A \textbf{3rd Place Certificate/Trophy} is then awarded.
After reminding the 1st place team to be respectful, a
\textbf{2nd Place Certificate/Trophy} is then awarded, followed by the
\textbf{1st Place Certificate/Trophy}.

Alternatively, \textbf{Certificates of Participation} vs \textbf{Completion}
might be distributed depending on whether each team completed the
story of the event by solving the final metapuzzle.

Opportunities for photographs might
be allowed during this process and after dismissal.
After recognitions are done, teams may be dismissed.

\phSection{Social Media}

Players/teachers/volunteers can be encouraged to tag \texttt{@MaPPmath}
on Twitter with non-spoiler posts/media during
and after the event. Everyone should be reminded that the game is played
at multiple locations across the country, and we do not want to spoil
future players on what's in store.

